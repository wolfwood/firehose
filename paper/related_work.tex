\documentclass[conference]{IEEEtran}
%\usepackage{cite}
\usepackage[pdftex]{graphicx}
%\graphicspath{{images/}}
\DeclareGraphicsExtensions{.pdf}
\usepackage{array}
\usepackage{url}
\usepackage{amsmath}

% correct bad hyphenation here
%\hyphenation{}

\begin{document}
\title{Using Interactive Protocols to Gain Power Savings in Wireless Communication}

\author{\IEEEauthorblockN{Nicholas Farnan and James Larkby-Lahet}
\IEEEauthorblockA{Department of Computer Science, University of Pittsburgh\\ Pittsburgh, PA 15260 -- USA \\ Email: \{nlf4,jamesll\}@cs.pitt.edu}}
\maketitle

\begin{abstract}
mah abstract, it r here
\end{abstract}

\section{Introduction}

i also has an introduction

\section{Background}



\section{Related Work}

In \cite{Adler98}, Adler and Maggs propose that asymmetric
communication channels could be exploited for the purposes of saving
power in wireless transmission.  Their communication model consists of
a client that wishes to send some data drawn from a probability
distribution $D$ to a server, where the server has knowledge of $D$
which the client lacks.  The server uses the asymmetry of
communication channels to send information about D to the client whose
responses can not only effectively communicate the intended data, but
also do so efficiently, minimizing the amount of data actually sent
from the client to the server. They propose protocols that will result
in an expectation of O($n$) bits being sent by the server and
O(H($D$)$ + 1$) bits being sent by the client where $n$ is the number
of bits in the data the client wishes to communicate with the server
and H($D$) being the binary entropy of $D$.  They offer two varieties
of such protocols, one where data is exchanged over some number of
rounds, and another where all communication occur in a single round at
the cost of additional server-side computation.  In both cases, one
round involves a single communication from server to client and
another single response.\\

This work was followed up by \cite{Watkinson01}, in which Watkinson,
Adler, and Fich provide protocols that result in H($D$)$ + 2$ bits
sent by the client in expectation and $n$(H($D$)$ + 2$)) bits sent by
the server in expectation.  Similar to \cite{Adler98}, the authors
provide both multi-round and single-round variations of their
protocol.  Again, the single-round variant comes at the cost of
additional server-side computation.\\

All of the above protocols were supplanted by Gagie's work in
\cite{Gagie06} where he showed that any dynamic instantaneous
compression algorithm could be converted to an asymmetric
communication protocol.  The intuition behind this is that Shannon's
theory requires the client to send H(D) bits, which is achieved by the
entropic encoding.  The only relevant information that the server may possess
that the client does not is the distribution observed across many
messages, including those from nodes other than the client.  By
transmitting this distribution, the dynamic instantaneous (one-pass)
encoder starts with a more representative distribution than the
uniform distribution it initially would assume.\\

Gagie's model of communication also differs from those previously
proposed in that he does not require the server to have pre-existing
knowledge of $D$ and instead accounts for the server aquire such
knowledge on-line as it receives messages from clients.

The work presented here is an implementation of Gagie's proposed
protocol.  It differs from Gagie's work in \cite{Gagie06} in that we
provide experimental results showing the effectiveness of using
dynamic instantaneous compression algorithms to reduce the amount of
data a client sends in order to communicate a message, as well as the
power savings that can be gleaned from the use of such a protocol.

\section{Implementation}

%Our implementation consists of an one pass range encoding compression algorithm that depends on a frequency distribution over all symbols in the alphabet of the message being sent.  This distribution is supplied to the client by the server.  The server begins with a flat model of equivalent frequencies for all symbols and updates this ditribution as

\bibliographystyle{IEEEtran}
\bibliography{firehose}
\end{document}
