
\section{Abstract}
{\it

//}

\section{Introduction}
Wireless communication devices are becoming commonplace across many
different application domains.  The characteristics of phone, wifi,
sensor and bluetooth networks and devices are quite diverse, however
several commonalities exist.\\  

Power concerns are common in wireless networks, as we frequently wish
to cut power cords as well as Ethernet cables.  Wireless nodes are
force to rely on limited battery capacity to provide the maximum
length of availability.  Recharging or replacing batteries ranges from
difficult or impossible in remote sensor network deployments, such as
jungle and marshland environmental monitoring, to major inconvenience in
the case of of personal electronics in the hands of a traveler.{CITE
  SOMETHING FOR THE LOVE OF GOD}\\

One interesting property of wireless networks that is often exploited
for wireless power management is asymmetry in the power levels of
various network nodes {\em XXX: CITE}.  At one level, some nodes,
frequently including routers to external networks, are connected to
power mains and do not rely on battery power.  These nodes may be able
to spend more energy in order to reduce energy expenditure in battery
powered nodes, for example by serving as routers for messages between
nodes.\\

In a cooperative setting, such as sensor networks, the locked-down
networks of cellular carriers or the collaborative WiFi mesh of the One
Laptop Per Child project's XO computer, the global usefulness of a
network can outweigh the individual power hoarding of nodes.  As many
sensor network nodes double as routers for other nodes' data, it is
important to keep as many nodes alive as possible.  In this setting
nodes may develop individual variations in power level, due to variation
in sensing and particularly routing workload.  We may wish to spend
more power on nodes with extra battery life if we can extend the
battery life of more significantly drained nodes.\\

It is natural that power-aware ad-hoc routing protocols have been
studied to reduce variability in node power levels thereby extending
the amount of time the network operates at maximum node strength,
perhaps at the cost of network life as node batteries become fully
drained, and the chances of network partition increase dramatically.
These protocols adjust the optional transmission data.  While messages
certainly must be sent, individual nodes do not necessarily need to
participate, and the choice of participants is the output of the
algorithm.\\

For a given node that is required to transmit data, these algorithms
can reduce the energy cost of transmission, however their sole
mechanism for doing so is the selection of a closer communication
partner for reduced transmission power, and hence energy
consumption.\\

Given the necessity of a communication in a wireless power-constrained
network, two cases allowing for energy optimization may exist.  We
aimed to address the case where a power-poor node must communicate
with an less energy-constrained receiver.  Specifically, we
investigated the potential for energy savings via interactive
communication, where the receiver may also transmit data to the sender
in an attempt to reduce the number of bits that must be sent, and
hence total energy consumed by the sender.  This is only possible if
the energy required to receive the additional bits is less than the
energy that would have been spent to transmit the additional bits.  We
demonstrate that this is indeed the case for a variety of real world
devices.\\

In the second case, an energy-rich sender node may attempt to reduce the
energy used by a energy-poor receiver by compressing data or using a
higher transmission rate to reduce the amount of time the receiver
spends receiving data.  There is no incentive to transmit data from
the receiver in this case as it can only increase the amount of energy
consumed; hence we do not address this case with this technique.\\

The remained of this paper is structured as follows, Section 2
presents background on the power characteristics of modern wireless
and sensor devices.  Section 3 presents related work.  Section 4
describes our approach.  In Section 5 we present and analyze experimental
results and conclude with Section 6.\\
